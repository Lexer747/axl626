\documentclass[12pt]{article}

%------------ PACKAGES ----------
\usepackage{mathtools}
\usepackage{csquotes}

%------------ MARGINS ----------
\addtolength{\oddsidemargin}{-.875in}
\addtolength{\evensidemargin}{-.875in}
\addtolength{\textwidth}{1.75in}
\addtolength{\topmargin}{-.875in}
\addtolength{\textheight}{1.75in}

%------------ TITLE INFO ------------
\title{Literature Review}
\author{Alex Lewis}
\date{\today}

%------------ START DOCUMENT ----------
\begin{document}
\maketitle
\section{Kelly Criterion}

    Kelly criterion is a well known concept for the most optimal way to grow a budget from a 
    given gamble. In the case where there is a positive expectation gamble, then using the 
    Kelly criterion is mathematically proven to be optimal. Based on the paper which under
    the guise of networking formally defines the optimal growth rate of a betting function
    with a fixed probability and payout.

    \begin{equation}\label{eq:KellyOriginal}
        k = \frac{(b \times p - (1 - p))}{b} 
    \end{equation}

    where:
    \begin{flalign*}
        k &= \text{The best size as a fraction of your portfolio} &&\\
        b &= \text{The net odds on the wager. i.e. your win } b + wager staked &&\\
        p &= \text{The probability of winning} &&
    \end{flalign*}

    The equation \ref{eq:KellyOriginal} is from the famous paper by Kelly in 1956 \cite{Kelly}.
    Much work has been done to adapt the formula and make it more relevant to current economics.
    Firstly the equation \ref{eq:KellyOriginal} has one clear difference to real stocks and 
    investment which is that you are assumed to lose 100\% of your wager upon a loss. Which 
    while technically true in the worst case for a stock or bond investment. Is more often than 
    not, not what happens. Typically you lose a smaller percentage than 100\%. Hence there is
    an adapted formula: {(From the article by Alon\cite{Alon})}

    \begin{equation}\label{eq:KellyWinLoss}
        k = \frac{p - (1 - p)}{\frac{W}{L}}
    \end{equation}

    where:
    \begin{flalign*}
        k &= \text{The best size as a fraction of your portfolio} &&\\
        p &= \text{The probability of winning} &&\\
        W &= \text{Is the amount of winnings} &&\\
        L &= \text{Is the amount of losses} &&
    \end{flalign*}

\subsection{Example}

    Say we have a gamble \(X\) with the following properties:

    \[P(X) = 0.6\]

    And \(X\) pays out \(1.2\times\) the stake. Win 20\%. And in the case \(1 - P(X) = 0.4\)
    \(X\) only takes \(0.8\times\) of the stake, instead of all of it. Lose 20\%. It should
    be apparent that \(X\) is a good investment. Heres its expected value in case you had
    doubts:

    \[E(X) = 0.6 \times 1.2 + 0.4 \times 0.8 = 1.04\]

    Hence the expected value of \(X\) is a return of 4\%. And if plug that into the Kelly
    formula, it will tell us how much of portfolio we should bet to see optimal growth.
    Subbing in our values into equation \ref{eq:KellyWinLoss}:

    \[k = \frac{0.6 - (1 - 0.6)}{\frac{0.2}{0.2}} = 0.2\]

    Hence we should bet 20\% of our portfolio on the next trade. The problem with this approach 
    is that its a ratio. So if our gamble is more volatile (or equally any stock) and a win is 
    200\% and a loss is 200\%, then the ratio between wins and losses is the same. But even
    though the investment is \(10\times\) more volatile, it still has the same kelly value:

    \[k = \frac{0.6 - (1 - 0.6)}{\frac{2}{2}} = 0.2\]

\subsection{The Correct Kelly Formula}

    \begin{equation}\label{eq:KellyCorrect}
        k = \frac{p}{L} - \frac{1 - p}{W}
    \end{equation}

    Equation \ref{eq:KellyCorrect} is also from Alon\cite{Alon}. As you can see this no longer 
    has the ratio of wins and losses as the denominator, so substituting into equation 
    \ref{eq:KellyCorrect}:

    \[k = \frac{0.6}{0.2} - \frac{1 - 0.6}{0.2} = 1\]

    As you can see, this Kelly value says to put 100\% of your portfolio on the bet. Which
    might seem a bit ridiculous but this is why people generally use a fraction of Kelly
    value, as even though it is mathematically the fastest way to grow a portfolio from a
    given gamble. It can be daunting to invest that much.

    \[k = \frac{0.6}{2} - \frac{1 - 0.6}{2} = 0.1\]

    But as we can see from the above calculation with a Win of 200\% and a loss of 200\%, it 
    correctly adjusts the bet size so that only 10\% of the portfolio is invested. Whereas the 
    other Kelly formula would not to change the bet size.

\section{Kelly Criterion Application}

    Kelly Criterion as concept is a great theoretical achievement, such a simple formula so
    perfectly describes how to play a system in order to gain the most from it. But alas
    stocks and trades are themselves not a repeatable and positive mathematical expectation
    game. Hence why we need to adapt the equations so that we can try to apply them to the
    real world.

\subsection{Optimal F}

    \begin{displayquote}
        'For any given independent trials situation where you have an edge (i.e. positive 
        expectation, see Kelly Criterion for an example) there exists an optimal fixed fraction (
        f) between 0 and 1 as a divisor of your biggest loss to bet on each event.'\cite{Ralph}
    \end{displayquote}

    As it turns out this optimal f is similar to what Kelly describes in his paper. And as Ralph 
    proceeds, he explains how Kelly criterion is a perfect solution for fixed size gambles with 
    fixed wins and losses.

    But then the book reaches the a different conclusion, it goes on to state that trades where 
    the win or loss is always changing {(like the stock market)} then Kelly formula does find the 
    correct optimal f. This lends to reason as it is a dream like world where we have a fixed 
    unchanging trade with a set win loss ratio.

    So instead the book proposes finding the optimal f by instead using the geometric mean. We 
    can use the estimated geometric mean because it is basically the same, while being much less 
    computation:

    \begin{equation}\label{eq:TWR}
        TWR = \displaystyle\prod^{N}_{i=1}1 + f \times \frac{- trade_i}{biggest loss}
    \end{equation}
    \begin{equation}\label{eq:GeoMean}
        Geometric Mean = exp(\frac{1}{N} \times Ln(TWR))
    \end{equation}

    where:
    \begin{flalign*}
    f &= \text{The optimal f} &&\\
    N &= \text{The number of trades} &&\\
    trade_i &= \text{The profit loss of the} i^{th} \text{trade from the set of trades} &&\\
    biggest loss &= \text{The biggest loss from the set of trades} &&\\
    exp(x) &= \text{The exponential function} &&\\
    Ln(x) &= \text{The natural logarithm function} &&
    \end{flalign*}

    Equation from Ralph Vince's book\cite{Ralph}.
    To maximize our profit from a set of trades we want to optimize for the highest possible 
    geometric mean. Since f is a free standing variable which cannot be made the subject of the 
    equation we can only use iteration to find a good estimate.

    \subsubsection{Example of Finding the optimal F}

    Given we trade \(N\) stocks buy buying them at point X and selling them at point Y we have a 
    sample trade sequence:

    \begin{center}
    \begin{tabular}{ |c|c|c|c| } 
     \hline
        \(N\) & X & Y & profit \\
        \hline
        1 & 50 & 59 & 9   \\
        2 & 50 & 68 & 18  \\
        3 & 50 & 57 & 7   \\
        4 & 50 & 51 & 1   \\
        5 & 50 & 60 & 10  \\
        6 & 50 & 45 & -5  \\
        7 & 50 & 47 & -3  \\
        8 & 50 & 33 & -17 \\
     \hline
    \end{tabular}
    \end{center}

    Note how the loss profit is different on each trade, hence Kelly's fraction to bet would be 
    incorrect. Now calculation with an arbitrary f of 0.1.

    Using the equation \ref{eq:TWR}

    \begin{center}
    \begin{tabular}{ |c|c|c|c| } 
     \hline
        \(N\) & \(i\) & TWR \\
        \hline
        1 & 1 + 0.1 * (-9/-17)  & 1.0529 \\
        2 & 1 + 0.1 * (-18/-17) & 1.1059 \\
        3 & 1 + 0.1 * (-7/-17)  & 1.0412 \\
        4 & 1 + 0.1 * (-1/-17)  & 1.0059 \\
        5 & 1 + 0.1 * (-10/-17) & 1.0588 \\
        6 & 1 + 0.1 * (5/-17)   & 0.9706 \\
        7 & 1 + 0.1 * (3/-17)   & 0.9824 \\
        8 & 1 + 0.1 * (17/-17)  & 0.9000 \\
        \hline
        & & \(\prod = 1.1081\)\\
     \hline
    \end{tabular}
    \end{center}

    Note: Since the geometric mean is exponential dependent on the \(TWR\), we can just skip that
    step.

    But this does not tell us the optimal fraction to bet, we simply chose a random fraction 
    0.1, so we should now calculate the mean with a different f; say 0.9.

    \begin{center}
    \begin{tabular}{ |c|c|c|c| } 
     \hline
        \(N\) & \(i\) & TWR \\
        \hline
        1 & 1 + 0.9 * (-9/-17)  & 1.4764 \\
        2 & 1 + 0.9 * (-18/-17) & 1.9529 \\
        3 & 1 + 0.9 * (-7/-17)  & 1.3705 \\
        4 & 1 + 0.9 * (-1/-17)  & 1.0529 \\
        5 & 1 + 0.9 * (-10/-17) & 1.5294 \\
        6 & 1 + 0.9 * (5/-17)   & 0.7352 \\
        7 & 1 + 0.9 * (3/-17)   & 0.8411 \\
        8 & 1 + 0.9 * (17/-17)  & 0.0099 \\
        \hline
        & & \(\prod = 0.3936\)\\
     \hline
    \end{tabular}
    \end{center}

    As we can see the \(TWR\) is much lower so we know that it is a worse f to use for betting. Hence
    we know a better f exists inside the 2 bounds. Solving for f by iteration:

    Optimal \(f = 0.237\)

    as the \(TWR = 1.212\)

\section{Portfolio Diversification}

    Finding the optimal f, or the fraction to bet is only a small part of actually determining 
    how to spread out a bank roll over multiple markets. At the end of day portfolio allocation 
    is about the mathematics of a system, rather than a single trade or bet. Consider a 2:1
    coin game which has favorable mathematically expectation, the optimal f is 0.25.

    But consider simultaneously playing 2 games, {(for simplicity another 2:1 coin game)} the 
    optimal f becomes a function of correlation between both games. As if these games are not 
    independent events, much like stocks on the market are not. Then the outcome of the first 
    game affecting the second game, or vice versa. Will affect the optimal f.

    Thinking of the optimal f as a single point, inside a space of trades/bets. When we have 
    only a single trade/bet, we are doing calculations in a 2D space, and we have a line which 
    represents out optimal f. However as the number of trades/bets increases we gain another 
    dimension for each one.

    Therefore instead a f which lies on a line, it lies on a plane, or a 3D object, and etc. If 
    the f suddenly becomes a multi variable coordinate which must be exactly correct. If a 
    single axis is out then you can miss the hill of positive growth, even if every other axis 
    lined up.

    This means we need to define a new function to find the individual f's for all the bets,
    but with relation to each other.

\subsection{Mathematical Definition}

    \begin{equation}\label{eq:G}
        G(f_1...f_n) = \left( \displaystyle\prod^{m}_{k=1} HPR_k \right) ^{ \left( \displaystyle\frac{1}{\sum^{m}_{k=1}Prob_k} \right)}
    \end{equation}
    \begin{equation}\label{eq:HPR_k}
        HPR_k = \left( 1 +  \displaystyle\sum^{n}_{i=1} f_k \times \frac{- PL_{k,i}}{BL_k} \right) ^{Prob_k}
    \end{equation}
    \begin{equation}\label{eq:Prob_k}
        Prob_k = \left( \displaystyle\prod^{n - 1}_{i=1} P(i_k | j_k)\right)^{\frac{1}{n - 1}}
    \end{equation}

    where:
    \begin{flalign*}
    n &= \text{The number of trades or bets in the } k^{th}\text{set} &&\\
    m &= \text{The number of combinations for all trades and bets, i.e. for each } n \text{ which has }x \text{ outcomes then } &&\\
    & \Big( m = \prod^{n}_{i=1} x_i \text{ : }(x \text{ is 2 for a bet, win and lose. Or 1 for a trade, the value gained}) \Big) &&\\
    f_k &= \text{The optimal } f \text{ for that }k^{th} \text{ set, where } f > 0 &&\\
    PL_{k,i} &= \text{The outcome for the } i^{th} \text{ trade or bet associated with the } k^{th} \text{ set} &&\\
    BL_k &= \text{The worst trade or bet for the } k^{th} \text{ set} &&\\
    P(i_k | j_k) &= \text{Roughly it is the risk of }i^{th} \text{ trade or bet associated with the } k^{th} \text{ set given the risk of the } &&\\
    & j^{th} \text{ trade or bet associated with the } k^{th} \text{ set. Which is easy to calculate for coins, but I will explain} &&\\
    & \text{later how it is calculated for actual stocks} &&
    \end{flalign*}

    This equation is also from Ralph's Book \cite{Ralph}.

\subsubsection{Example applying the Mathematical Formula}

    Applying \(G\) to a setting in which we actually want to optimize how we bet our 
    portfolio. Say we have 3 coins, Coin 1, Coin 2, Coin 3. Each one has 2 outcomes, and for 
    simplicity we assume that the correlation coefficients are zero. I.e. each coin is
    independent. And modeling the risk more simply than a stock.

    This means we need to find 3 f values \(f_1, f_2, f_3\) to result in the greatest growth.
    Or the  3 f's which result in the greatest \(G\).

    Re-writing \(HPR_k\) to:

    \begin{align*}
    HPR_k &= (1 + C)^{Prob_k} \\
    C &= \displaystyle\sum^{n}_{i=1}f_i \times (- PL_{k,i} / BL_i)
    \end{align*}

    We can calculate \(C\). Given that \(BL\) is the same for each coin, say -1, and the \(PL\) 
    is -1. I.e. when we lose we gain -£1, when we loss our biggest loss is -£1. This is because 
    \(k=1\) so this the first outcome, where it lands tails. If we were to win then \(PL\)
    would be 2, as we are playing a 2:1 game.

    Once again we have to use \(f_i\) without knowing what it is, so pick a random start of
    0.1. And \(n\) is 3 since we have 3 coins.

    \begin{flalign*}
    C &= (0.1 * (- -1 / -1)) + (0.1 * (- -1 / -1)) + (0.1 * (- -1 / -1)) &&\\
    C &= (0.1 * -1) + (0.1 * -1) + (0.1 * -1) &&\\
    C &= -0.3 &&\\
    HPR_k &= (1 + -0.3)^{Prob_k} &&\\
    HPR_k &= (0.7)^{Prob_k} &&
    \end{flalign*}

    Now we need to calculate equation \ref{eq:Prob_k} so we can find \(Prob_k\). And I'll
    emphasize again, this calculation will be a simplification of what we will actually do
    with a real stock, since we are dealing with coins in this example.

    \begin{center}
    \begin{tabular}{ |c|c|c|c| } 
     \hline
        \(i\) & \(j\) & \(k\) & \(P(i_k | j_k)\) \\
        \hline
        1 & 2 & 1 & \(P(0.5 \cap 0.5) = 0.25\)\\
        1 & 3 & 1 & \(P(0.5 \cap 0.5) = 0.25\)\\
        2 & 3 & 1 & \(P(0.5 \cap 0.5) = 0.25\)\\
        \hline
        & & & \(\prod = 0.015625\)\\
     \hline
    \end{tabular}
    \end{center}

    \begin{flalign*}
    Prob_k &= 0.015625^{\frac{1}{3 - 1}} &&\\
    Prob_k &= 0.015625^{0.5} &&\\
    Prob_k &= 0.125 &&
    \end{flalign*}

    Now we can calculate \(HPR_k\) where \(k = 1\):

    \begin{flalign*}
    HPR_k &= (1 + C)^{Prob_k} &&\\
    HPR_k &= (0.7)^{0.125} &&\\
    HPR_k &= 0.95639490757 &&
    \end{flalign*}

    Hence to find \(G\) we need to find all the values of \(HPR_k\):

    \begin{center}
    \begin{tabular}{ |c|c|c|c| } 
     \hline
        \(k\) & \(C\) & \(Prob_k\) & \(HPR_k\) \\
        \hline
        1 & -0.3 & 0.125 & 0.9564 \\
        2 & 0.0  & 0.125 & 1.0333 \\
        3 & 0.0  & 0.125 & 1.0000 \\
        4 & 0.3  & 0.125 & 1.0000 \\
        5 & 0.0  & 0.125 & 1.0333 \\
        6 & 0.3  & 0.125 & 1.0000 \\
        7 & 0.3  & 0.125 & 1.0333 \\
        8 & 0.6  & 0.125 & 1.0605 \\
     \hline
    \end{tabular}
    \end{center}

    Subbing the values into \(G\):

    \begin{flalign*}
    G(0.1, 0.1, 0.1) &= \left( \displaystyle\prod^{m}_{k=1} HPR_k \right) ^{ \left( \displaystyle\frac{1}{\sum^{m}_{k=1}Prob_k} \right)} &&\\
    G(0.1, 0.1, 0.1) &= (0.9564 \times 1 \times 1 \times 1.0333 \times 1 \times 1.0333 \times 1.0333 \times 1.0605 ) ^{ \left( \displaystyle\frac{1}{\sum^{m}_{k=1}Prob_k} \right)} &&\\
    G(0.1, 0.1, 0.1) &= (1.1190) ^{ \left( \displaystyle\frac{1}{(0.125 + 0.125 + 0.125 + 0.125 + 0.125 + 0.125 + 0.125 + 0.125)} \right)} &&\\
    G(0.1, 0.1, 0.1) &= (1.1190) ^{ \left( \displaystyle\frac{1}{(0.125 + 0.125 + 0.125 + 0.125 + 0.125 + 0.125 + 0.125 + 0.125)} \right)} &&\\
    G(0.1, 0.1, 0.1) &= (1.1190) ^{1} &&\\
    G(0.1, 0.1, 0.1) &= 1.1190 &&
    \end{flalign*}

    Now to finish we should iterate over the set of \(f_i\) to find the largest \(G\), but after
    inspection we can see to iterate over all values of \(f\) would scale incredibly badly
    the number of calculations goes up far too fast.

    And if you've been anywhere near AI for past few years, optimizing \(f\) for the largest 
    \(G\) is a classical problem which is often solved by neural networks or genetic
    algorithms. Its gradient descent. Given an array of inputs, and a evaluation function,
    find the best array of inputs to score the best in the evaluation function. In our case,
    the array is \(f\), and our function is \(G\).

    But for completeness Ralph\cite{Ralph} shows how if you do find the best \(f\) for 3 coin
    games, its different than just applying Kelly to each game individually. And the \(f\)
    chosen by this method outperforms the basic Kelly formula.

\section{Calculating Risk}

    Here we are actually constructing the risk calculation for a stock in a more concrete form.
    So must operate under some assumptions and natural patterns of a stock. 

    Proposed Model of a Stock: 
    \begin{itemize}
        \item{Open {\&} Close values for a given arbitrary time period}
        \item{A ratio of correlation to other stocks; see Ralph \cite{Ralph} for reasoning}
        \item{Holes of Open {\&} Close values may occur}
    \end{itemize}

    Calculating the profit{\/}loss of a stock at time period is now trivial and useful.
    Renaming the change in value at each time period \(i\) to \(PL_i\) we can find
    statistical facts about the stock.

    \begin{align}
        \bar{PL} &= \frac{\sum^{i}_{n=1} PL_n}{i} \label{eq:StockMean} \\
        \sigma^2 &= \frac{\sum^{i}_{n=1} (PL_n - \bar{PL})^2}{i} \label{eq:StockVar}
    \end{align}

    Using this data we can transform a stock into a distribution which we can use to estimate
    the likely hood of the next \(PL_i\) being above a certain value. In our case we transform
    the normal distribution using equations \ref{eq:StockMean} and \ref{eq:StockVar} to find
    the mean and variance of the stock, and hence change the normal distribution. Then it is
    simple enough to use tables to find probabilities we want.

    Accounting for holes in the data, and the fact that older data may not be as relevant. Is
    Currently overlooked if we were just going to

    simulated annealing 
    

\section{Machine Learning And Finding \(G\)}

    Equations \ref{eq:G}, \ref{eq:HPR_k}, and \ref{eq:Prob_k} can be combined into one equation:

    \begin{equation}\label{eq:FullG}
        G(f_1...f_n) = \left( \displaystyle\prod^{m}_{k=1} \Bigg( 1 + \displaystyle\sum^{n}_{i=1} f_k \times \Big( \frac{- PL_{k,i} }{BL_k}\Big) \Bigg)^{\Bigg( \displaystyle\prod^{n - 1}_{i=1}  P(i_k | j_k) \Bigg) ^ {\frac{1}{n - 1}}} \right) ^ {\left( {1 \div {\displaystyle\sum^{m}_{k=1} \Bigg( \displaystyle\prod^{n - 1}_{i=1}  P(i_k | j_k) \Bigg) ^ {\frac{1}{n - 1}}}} \right)}
    \end{equation}

    where:
    \begin{flalign*}
    n &= \text{The number of trades or bets in the } k^{th}\text{set} &&\\
    m &= \text{The number of combinations for all trades and bets} &&\\
    f_k &= \text{The optimal } f \text{ for that }k^{th} \text{set, where } f > 0 &&\\
    PL_{k,i} &= \text{The outcome for the } i^{th} \text{ trade or bet associated with the } k^{th} \text{ set} &&\\
    BL_k &= \text{The worst trade or bet for the } k^{th} \text{ set} &&\\
    P(i_k | j_k) &= \text{The risk of }i^{th} \text{ trade or bet associated with the } k^{th} \text{ set given the risk of the } &&\\
    & j^{th} \text{ trade or bet associated with the } k^{th} \text{ set.} &&
    \end{flalign*}

    Inside this function we actually have 3 other functions which are needed to compute \(G\).
    As described \(PL_{k,i}\), \(BL_k\) and \(P(i_k | j_k)\) are all their own functions, which
    use stocks, and other variables to be calculated. This is not an issue when it comes to
    calculating the value of \(G\) as we know all these variables. But it becomes an issue if
    want to differentiate \(G\) with respect to \(f\). I believe it would be possible to
    differentiate this equation using the chain rule, but then it becomes differentiated with
    respect to a single \(f\).

\section{Portfolio Optimization}

    This model of \(G\) is just one method, another model is suggested by Zachariah
    \cite{Zachariah} in which the author combines the same model proposed above, with the 
    addition of a risk model in which each trade or bet when combined forms a linear correlation 
    using an estimation technique which can then be used to find the variance of the system.

    I shall leave the specific implementation of this model to the reader if they wish to 
    understand the latest and greatest models currently in use. This model also uses genetic 
    algorithms to solve the model for the set of f. Since once again its a evaluation function, 
    with a large set of inputs which can adjusted to find the optimum set of inputs which best 
    performs according to the function.

    But Later in my report I discuss my approach to handling risk, which goes in a different
    direction to Zachariah.


\section{Random Walks}

    There is a competing train of the thought to the one described by the narrative above. And 
    even though it appears to have a large theoretical, and practical support, it is still an 
    ongoing debate among economists as to which train of thought is better or even if either is 
    correct.

    As described in the book 'A Random Walk Down Wall Street' by Burton \cite{Burton} it uses
    key economic historical events and many studies to describe how investing and trading are
    a futile game, compared to compound interest, or using an Index like S\&P 500. Which
    simply is a summation of the top 500 stocks in the US stock market.

    Instead of applying mathematics and statistics to your stock purchasing its contrary the 
    very nature of randomness, which is that humans look for patterns in all things. True 
    randomness seems to be against the very nature of what it is to be human, which is why
    there is still contention as to the ideas in the book. The idea that if you examine stock
    prices in a given period you can always find a system that would produce a positive return.

    A system which will always work and produce a positive return, and be mathematically sound. 
    I.e. it will work infinitely. Is an impossibility. We do not have an infinite amount of
    time to test our systems on infinite data.

    However, as much as I like the argument presented in the book and the complete contrast to 
    statistically minded way to analyze stocks. Ultimately a stock is buying a piece of an 
    actual business something which is valued on how it provides a product or service. And 
    looking back at businesses it becomes easier to see how the value of its stock is directly 
    related to the people in it, manning the ship.

    And for me, the only way to me to reason to myself that statistics, and analysis are still 
    useful. Is that a market is a system of people, which is something I don't believe we will 
    ever fully understand. But, statistics and short term analysis can give us insights as to 
    how it might work. It's too easy - for me at least - to accept that we our powerless and 
    nothing we can do in the long run will change that we can't beat randomness. Hence why
    trying to 'game' the system by using short term analysis to find businesses which happen
    to be doing well right now, for short term returns, is the only option we have against
    randomness.

\pagebreak
\bibliography{LiteratureReview}
\bibliographystyle{plain}

\end{document}